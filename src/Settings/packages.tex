\usepackage{polyglossia} % языковой пакет

\usepackage{pdfpages} % пакет для импорта pdf-файлов

\usepackage{tocvsec2} %%%%%%%%%%%%%%%%%%%%%%%%%%%%%%%%%

\usepackage{longtable,booktabs,array}

\usepackage{calc}

\usepackage{ulem}

\usepackage{setspace}

\usepackage[labelsep=period]{caption}

\usepackage{caption}

\usepackage{graphicx} % пакет для использования графики (чтобы вставлять рисунки, фотографии и пр.)


% настройки оглавления
\usepackage{tocloft}
\renewcommand{\cftpartleader}{\cftdotfill{\cftdotsep}} % for parts
\renewcommand{\cftsecleader}{\cftdotfill{\cftdotsep}} % for sections, if you really want! (It is default in report and book class (So you may not need it).


% качественные листинги кода
\usepackage{minted}
\usepackage{listings}
\usepackage{lstfiracode}

% Елочки-скобки с помощью /textquote
\usepackage{csquotes}

\usepackage{amsmath} % поддержка математических символов

\usepackage{url} % поддержка url-ссылок

\usepackage{natbib} % менеджер цитирования natlib.

\bibliographystyle{unsrtnat} % выбираем стиль библиографии отсюда: https://www.overleaf.com/learn/latex/Natbib_bibliography_styles

\setcitestyle{authoryear, open={[},close={]}} % Определяем стиль цитирования. Указываем, чтобы цитирование в тексте вставлялось в формате (Автор, год). 
% \bibpunct{[}{]}{,}{n}{}{,}

\usepackage{multirow} % таблицы с объединенными строками

\usepackage{hyperref} % пакет для интеграции гиперссылок

\usepackage{indentfirst} % пакет для отступа абзаца


\usepackage{chngcntr} % пакет подписей и нумерации рисунков



%%%%%%%%%%%%%%%%%%%%%%%%%%%%%%%%%%%%%%%%%%%%%%%%%%%%%%%%%%%%%%